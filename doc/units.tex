\documentclass[11pt]{amsart}
\usepackage{geometry}                % See geometry.pdf to learn the layout options. There are lots.
\geometry{letterpaper}                   % ... or a4paper or a5paper or ... 
%\geometry{landscape}                % Activate for for rotated page geometry
%\usepackage[parfill]{parskip}    % Activate to begin paragraphs with an empty line rather than an indent
\usepackage{graphicx}
\usepackage{amssymb}
\usepackage{epstopdf}
\usepackage{url}
\DeclareGraphicsRule{.tif}{png}{.png}{`convert #1 `dirname #1`/`basename #1 .tif`.png}

\title{Spectroscopic Units}
\author{Chris Stoughton}
\date{}                                           % Activate to display a given date or no date


\def\Var{\mathop{\rm Var}} 

\begin{document}
\maketitle

Spectra are expressed in various units:
\begin{itemize}
\item $F_\gamma$ =photons/cm$^2$/$\AA$/sec
\item $F_\lambda$ = ergs/cm$^2$/$\AA$/sec
\item $F_\nu$ = ergs/cm$^2$/Hz/sec
\item $F_{AB}$ =  Magnitude
\end{itemize}

We express spectra in $F_\gamma$.  The spectra we gathered for MKIDStd
may be expressed in other units.

For spectra expressed in $F_\lambda$, we use $E=h\nu$ and $c = \lambda\nu$:
\begin{equation}
F_\gamma = F_\lambda \lambda / hc
\end{equation}

The python package scipy.constants is a convenient way to get constants.
\begin{verbatim}
from scipy.constants import *
\end{verbatim}
defines h=6.62606957e-34 Joule$\cdot$sec and c=299792458.0
meters/sec.

For $\lambda$ expressed in $\AA$:

\begin{equation}
F_\gamma = F_\lambda \times \lambda(\AA) \times \frac{10^{-10} m/\AA}{ch10^7 erg/J} =  K \times F_\lambda \times \lambda(\AA)
\end{equation}
where $K=5.03E7$/erg/$\AA$. 

Some spectra are expressed in AB magnitudes.  Our good friends at
\url{http://en.wikipedia.org/wiki/AB_magnitude} tell us that
when flux is in erg/s/cm$^2$/$\AA$ the AB flux is
\begin{equation}
AB = -2.5\log_{10}{F_\lambda} -  5\log_{10}\lambda - 2.406
\end{equation}
Given an AB magnitude and $\lambda$ in $\AA$,
\begin{equation}
F_\lambda = (10^{-2.406/2.5})(10^{-0.4AB})/\lambda^2
\end{equation}




\end{document}
